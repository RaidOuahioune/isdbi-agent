\documentclass{article}%
\usepackage[T1]{fontenc}%
\usepackage[utf8]{inputenc}%
\usepackage{lmodern}%
\usepackage{textcomp}%
\usepackage{lastpage}%
%
%
%
\begin{document}%
\normalsize%
\section{AAOIFI Standard (Enhanced)}%
\label{sec:AAOIFIStandard(Enhanced)}%
\subsection{Clause 100.000}%
\label{subsec:Clause100.000}%
2 Billings by Islamic bank to (Al- Billings by Islamic bank to (Al- Mustasni’) purchaser (y-co.) Mustasni’) purchaser (y-co.)

%
\subsection{Clause 2/1}%
\label{subsec:Clause2/1}%
Istisna’a costs a) Istisna’a costs consist of: (I) Direct costs, in particular costs of  a) Istisna’a costs consist of: (I) Direct costs, in particular costs of  producing Al-Masnoo’; and (II) Indirect costs relating to the  producing Al-Masnoo’; and (II) Indirect costs relating to the  contract as allocated on an objective basis. General and admin- contract as allocated on an objective basis. General and admin- istrative expenses, selling expenses, research and development  istrative expenses, selling expenses, research and development  costs shall not be included in an Istisna’a contract costs. (para. 2) costs shall not be included in an Istisna’a contract costs. (para. 2) b) Istisna’a costs incurred during a financial period, as well as pre- b) Istisna’a costs incurred during a financial period, as well as pre- contract costs as described in (c) below, shall be recognized in an  contract costs as described in (c) below, shall be recognized in an  Istisna’a work-in-progress account, and reported under assets in  Istisna’a work-in-progress account, and reported under assets in  the statement of financial position of the Islamic bank. (In case of  the statement of financial position of the Islamic bank. (In case of  parallel Istisna’a, this account shall be called Istisna’a costs account  parallel Istisna’a, this account shall be called Istisna’a costs account  as stated in 2/2 a). Amounts billed to Al-Mustasni’ will be debited  as stated in 2/2 a). Amounts billed to Al-Mustasni’ will be debited  to Istisna’a accounts receivable account and credited to an Istisna’a  to Istisna’a accounts receivable account and credited to an Istisna’a  billings account. The balance of the latter account shall be offset  billings account. The balance of the latter account shall be offset  against Istisna’a work-in-progress account in the appropriate side  against Istisna’a work-in-progress account in the appropriate side  of the Islamic bank’s statement of financial position. (para. 3) of the Islamic bank’s statement of financial position. (para. 3) Financial Accounting Standard No. (10): Istisna’a and Parallel Istisna’a Financial Accounting Standard No. (10): Istisna’a and Parallel Istisna’a

%
\subsection{Clause 2/2}%
\label{subsec:Clause2/2}%
Contract costs in parallel Istisna’a a) When a parallel Istisna’a exists, the costs of Istisna’a include the  a) When a parallel Istisna’a exists, the costs of Istisna’a include the  price fixed in the parallel Istisna’a contract (direct costs), together  price fixed in the parallel Istisna’a contract (direct costs), together  with indirect costs including any pre-contract costs as described in  with indirect costs including any pre-contract costs as described in

%
\subsection{Clause 2/3}%
\label{subsec:Clause2/3}%
Istisna’a revenue and profit at the end of a financial period

%
\subsection{Clause 2/3/1}%
\label{subsec:Clause2/3/1}%
Istisna’a revenue and profit   Istisna’a revenue is the total price agreed upon between the    Istisna’a revenue is the total price agreed upon between the  Islamic bank as Al-Sani’ and the client as Al-Mustasni’, including  Islamic bank as Al-Sani’ and the client as Al-Mustasni’, including  the Islamic bank’s profit margin on the contract. Istisna’a revenue  the Islamic bank’s profit margin on the contract. Istisna’a revenue  and the associated profit margin are recognized in the Islamic  and the associated profit margin are recognized in the Islamic  bank’s financial statements according to either the percentage of  bank’s financial statements according to either the percentage of  completion or the completed contract methods as set up below,  completion or the completed contract methods as set up below,  taking into consideration what is stated in item 2/3/1/2. (para. 7) taking into consideration what is stated in item 2/3/1/2. (para. 7)

%
\subsection{Clause 2/3/1/1}%
\label{subsec:Clause2/3/1/1}%
Percentage of completion method a) A part of the contract price commensurate with the  a) A part of the contract price commensurate with the  work performed during each period in which the  work performed during each period in which the  Financial Accounting Standard No. (10): Istisna’a and Parallel Istisna’a Financial Accounting Standard No. (10): Istisna’a and Parallel Istisna’a

%
\subsection{Clause 2/3/1/2}%
\label{subsec:Clause2/3/1/2}%
Completed contract method   In unusual circumstances where both the percentage    In unusual circumstances where both the percentage  of completion and the expected cost to complete the  of completion and the expected cost to complete the  contract cannot be estimated with reasonable accuracy  contract cannot be estimated with reasonable accuracy  at the end of the financial period, no contract revenue  at the end of the financial period, no contract revenue  shall be recognized until the contract is fully completed.  shall be recognized until the contract is fully completed.  Thus, until that date, the accumulated contract costs  Thus, until that date, the accumulated contract costs  will be carried forward on the Istisna’a work-in-progress  will be carried forward on the Istisna’a work-in-progress  account, without any profit element being recognized.  account, without any profit element being recognized.  (para. 10) (para. 10)

%
\subsection{Clause 2/3/2}%
\label{subsec:Clause2/3/2}%
Deferred profits   The contract price may be fully paid by Al-Mustasni’ by    The contract price may be fully paid by Al-Mustasni’ by  instalments during the contract based on progress in work.  instalments during the contract based on progress in work.  However, all or part of the price may be paid following  However, all or part of the price may be paid following  completion of the contract. In the latter case, the difference  completion of the contract. In the latter case, the difference  between the total price that is paid during the contract and  between the total price that is paid during the contract and  the agreed total price –defined as deferred profits– shall be  the agreed total price –defined as deferred profits– shall be  offset against Istisna’a accounts receivable in the Islamic  offset against Istisna’a accounts receivable in the Islamic  bank’s statement of financial position. This treatment shall  bank’s statement of financial position. This treatment shall  Financial Accounting Standard No. (10): Istisna’a and Parallel Istisna’a Financial Accounting Standard No. (10): Istisna’a and Parallel Istisna’a

%
\subsection{Clause 2/3/3}%
\label{subsec:Clause2/3/3}%
Early settlement a) If Al-Mustasni’ makes a payment in advance of the due date  a) If Al-Mustasni’ makes a payment in advance of the due date  for such a payment, the Islamic bank may waive part of its  for such a payment, the Islamic bank may waive part of its  profit in recognition of this earlier payment. In that case,  profit in recognition of this earlier payment. In that case,  the amount of profit waived shall be deducted from both  the amount of profit waived shall be deducted from both  Istisna’a accounts receivable account and deferred profits  Istisna’a accounts receivable account and deferred profits  account. (para. 14) account. (para. 14) b) The same accounting treatment as in (a) above shall apply  b) The same accounting treatment as in (a) above shall apply  if the facts are the same except that the Islamic bank did not  if the facts are the same except that the Islamic bank did not  grant a partial reduction of the profit when the payment  grant a partial reduction of the profit when the payment  was made, but reimbursed Al-Mustasni’ with this amount  was made, but reimbursed Al-Mustasni’ with this amount  after receiving the payments. (para. 15) after receiving the payments. (para. 15)

%
\subsection{Clause 2/3/4}%
\label{subsec:Clause2/3/4}%
Parallel Istisna’a revenue and profit a) Parallel Istisna’a revenue and profit for each financial pe- a) Parallel Istisna’a revenue and profit for each financial pe- riod shall be measured and recognized according to the  riod shall be measured and recognized according to the  percentage of completion method, since in parallel Istisna’a  percentage of completion method, since in parallel Istisna’a  both costs and revenues of Istisna’a are known to the Islamic  both costs and revenues of Istisna’a are known to the Islamic  bank with reasonable certainty. (para. 16) bank with reasonable certainty. (para. 16) b) The recognized portion of Istisna’a profits for a financial  b) The recognized portion of Istisna’a profits for a financial  period shall be added to Istisna’a costs account. Thus, at any  period shall be added to Istisna’a costs account. Thus, at any  Financial Accounting Standard No. (10): Istisna’a and Parallel Istisna’a Financial Accounting Standard No. (10): Istisna’a and Parallel Istisna’a

%
\subsection{Clause 2/4}%
\label{subsec:Clause2/4}%
Measurement of Istisna’a work-in-progress, Istisna’a costs and treat- ment of contract losses at the end of a financial period ment of contract losses at the end of a financial period a) In the event of applying the percentage of completion method  a) In the event of applying the percentage of completion method  for the recognition of Istisna’a revenue and profit, Istisna’a work- for the recognition of Istisna’a revenue and profit, Istisna’a work- in-progress shall be measured and reported in the statement of  in-progress shall be measured and reported in the statement of  financial position of the Islamic bank at a value not exceeding its  financial position of the Islamic bank at a value not exceeding its  cash equivalent value (i.e., the difference between the contract  cash equivalent value (i.e., the difference between the contract  price and the expected additional cost to complete the contract).  price and the expected additional cost to complete the contract).  (para. 19) (para. 19) b) Any expected loss resulting from the valuation of Istisna’a work- b) Any expected loss resulting from the valuation of Istisna’a work- in-progress at the end of a financial period shall be recognized  in-progress at the end of a financial period shall be recognized  and reported in the Islamic bank’s income statement. (para. 20) and reported in the Islamic bank’s income statement. (para. 20) c) When a parallel Istisna’a exists, Istisna’a costs shall be treated as  c) When a parallel Istisna’a exists, Istisna’a costs shall be treated as  in (a) and (b) above. (para. 21) in (a) and (b) above. (para. 21) d) The subcontractor may fail to honour his obligation in a parallel  d) The subcontractor may fail to honour his obligation in a parallel  Istisna’a resulting in additional cost for the Islamic bank to fulfil its  Istisna’a resulting in additional cost for the Islamic bank to fulfil its  obligation towards Al-Mustasni’ (the client). Any such additional  obligation towards Al-Mustasni’ (the client). Any such additional  costs shall be recognized as losses in valuing the Istisna’a costs  costs shall be recognized as losses in valuing the Istisna’a costs  and reported in the Islamic bank’s income statement, except if  and reported in the Islamic bank’s income statement, except if  there is a reasonable degree of certainty that the Islamic bank  there is a reasonable degree of certainty that the Islamic bank  shall recover these additional costs. (para. 22) shall recover these additional costs. (para. 22)

%
\subsection{Clause 2/5}%
\label{subsec:Clause2/5}%
Change orders and additional claims a) The value and cost of change orders authorized by the Islamic bank  a) The value and cost of change orders authorized by the Islamic bank  and Al-Mustasni’ shall be added to Istisna’a revenue and costs,  and Al-Mustasni’ shall be added to Istisna’a revenue and costs,  respectively. (para. 23) respectively. (para. 23) Financial Accounting Standard No. (10): Istisna’a and Parallel Istisna’a Financial Accounting Standard No. (10): Istisna’a and Parallel Istisna’a

%
\subsection{Clause 2/6}%
\label{subsec:Clause2/6}%
Maintenance and warranty costs of Al-Masnoo’ a) Maintenance and warranty costs of Al-Masnoo’ shall be accounted  a) Maintenance and warranty costs of Al-Masnoo’ shall be accounted  for on an accrual basis. Such costs shall be estimated and then  for on an accrual basis. Such costs shall be estimated and then  matched with recognized Istisna’a revenue. Actual maintenance  matched with recognized Istisna’a revenue. Actual maintenance  and warranty expenditures shall be charged against a maintenance  and warranty expenditures shall be charged against a maintenance  and warranty allowance account when carried out by the Islamic  and warranty allowance account when carried out by the Islamic  bank. (para. 27) bank. (para. 27) b) When a parallel Istisna’a exists, the maintenance and warranty  b) When a parallel Istisna’a exists, the maintenance and warranty  cost of Al-Masnoo’ shall be accounted for on a cash basis, where  cost of Al-Masnoo’ shall be accounted for on a cash basis, where  such costs are charged by the Islamic bank directly to expense  such costs are charged by the Islamic bank directly to expense  accounts when they are incurred. (para. 28) accounts when they are incurred. (para. 28)

%
\subsection{Clause 230.000}%
\label{subsec:Clause230.000}%
230.000

%
\subsection{Clause 3/1}%
\label{subsec:Clause3/1}%
According to the Hanafis, Istisna’a’ ought to have been prohibited for  contravening the general Shari’a rules of  contravening the general Shari’a rules of Qiyas Qiyas (analogical deduction).   (analogical deduction).  They base their argument on the fact that the subject-matter of  They base their argument on the fact that the subject-matter of  a contract of sale ought to be in existence and in possession of the  a contract of sale ought to be in existence and in possession of the  seller, which is not the case in Istisna’a. The Hanafis have, nevertheless,  seller, which is not the case in Istisna’a. The Hanafis have, nevertheless,  approved the Istisna’a contract on the basis of  approved the Istisna’a contract on the basis of Istihsan Istihsan (juristic   (juristic  approbation) for the following reasons: approbation) for the following reasons: a) People have been practicing Istisna’a widely and continuously  a) People have been practicing Istisna’a widely and continuously  without condemnation, to the extent of furnishing a case of  without condemnation, to the extent of furnishing a case of Ijma’  Ijma’  (Consensus). (Consensus). b) It is possible in Shari’a to depart from  b) It is possible in Shari’a to depart from Qiyas Qiyas based on   based on Ijma’. Ijma’. c) The validity of Istisna’a is called for on grounds of need. People are  c) The validity of Istisna’a is called for on grounds of need. People are  often in need of commodities that are not available in the market,  often in need of commodities that are not available in the market,  and hence, they would tend to enter into contracts to have the goods  and hence, they would tend to enter into contracts to have the goods  manufactured for them. manufactured for them.(9) (9)

%
\subsection{Clause 3/2}%
\label{subsec:Clause3/2}%
Istisna’a is also valid in accordance with the general rule of the  permissibility of contracts as long as this does not contravene any  permissibility of contracts as long as this does not contravene any  text or rule of Shari’a. text or rule of Shari’a.(2) (2)

%
\subsection{Clause 3/2/1}%
\label{subsec:Clause3/2/1}%
Receipt of Al-Masnoo’ in conformity with specifications and  schedule schedule a) The received (Al-Masnoo’) assets shall be recorded at histo- a) The received (Al-Masnoo’) assets shall be recorded at histo- rical cost (i.e., the book value) of the Istisna’a costs account.  rical cost (i.e., the book value) of the Istisna’a costs account.  (para. 31) (para. 31) b) When a parallel Istisna’a exists, and Al-Masnoo’ is delivered  b) When a parallel Istisna’a exists, and Al-Masnoo’ is delivered  to the Al-Mustasni’ the balance of the Istisna’a costs account  to the Al-Mustasni’ the balance of the Istisna’a costs account  shall be transferred to an asset account that reflects the  shall be transferred to an asset account that reflects the  nature of Al-Masnoo’ received. (para. 32) nature of Al-Masnoo’ received. (para. 32)

%
\subsection{Clause 3/2/2}%
\label{subsec:Clause3/2/2}%
Late delivery of Al-Masnoo’   If the delay in the delivery of Al-Masnoo’ is due to the neg-   If the delay in the delivery of Al-Masnoo’ is due to the neg- ligence or fault of Al-Sani’ and the Islamic bank is entitled  ligence or fault of Al-Sani’ and the Islamic bank is entitled  to compensation for damages resulting from the delay, the  to compensation for damages resulting from the delay, the  amount of compensation shall be taken from performance  amount of compensation shall be taken from performance  bonds. If the amount of performance bonds is not sufficient  bonds. If the amount of performance bonds is not sufficient  to cover the amount of compensation, the balance shall be  to cover the amount of compensation, the balance shall be  recognized as Istisna’a accounts receivable due from Al-Sani’  recognized as Istisna’a accounts receivable due from Al-Sani’  and, if necessary, an allowance for doubtful debts account  and, if necessary, an allowance for doubtful debts account  shall be formed. (para. 33) shall be formed. (para. 33)

%
\subsection{Clause 3/2/3}%
\label{subsec:Clause3/2/3}%
Al-Masnoo’ not conforming to the specification a) If the Islamic bank declined to receive Al-Masnoo’ due  a) If the Islamic bank declined to receive Al-Masnoo’ due  to  to nonconformity to specifications and did not recover the  nonconformity to specifications and did not recover the  entire amount of progress payments made to Al-Sani’, the  entire amount of progress payments made to Al-Sani’, the  balance shall be recorded as Istisna’a accounts receivable,  balance shall be recorded as Istisna’a accounts receivable,  and if necessary, an allowance for doubtful debt account  and if necessary, an allowance for doubtful debt account  shall be formed. (para. 34) shall be formed. (para. 34) b) If the Islamic bank accepted Al-Masnoo’ which does not  b) If the Islamic bank accepted Al-Masnoo’ which does not  conform to specifications, such assets shall be measured at  conform to specifications, such assets shall be measured at  the lower of their cash equivalent value or historical cost  the lower of their cash equivalent value or historical cost  (the book value). Any resulting uncompensated loss shall  (the book value). Any resulting uncompensated loss shall  be recognized in the Islamic bank’s income statement for  be recognized in the Islamic bank’s income statement for  the current financial period. (para. 35) the current financial period. (para. 35) Financial Accounting Standard No. (10): Istisna’a and Parallel Istisna’a Financial Accounting Standard No. (10): Istisna’a and Parallel Istisna’a

%
\subsection{Clause 3/2/4}%
\label{subsec:Clause3/2/4}%
Al-Mustasni’ refuses to receive Al-Masnoo’   If Al-Mustasni’ (the client) refuses to receive Al-Masnoo’, the    If Al-Mustasni’ (the client) refuses to receive Al-Masnoo’, the  Istisna’a assets shall be measured at the lower of their cash  Istisna’a assets shall be measured at the lower of their cash  equivalent value or historical cost (the book value). Any resulting  equivalent value or historical cost (the book value). Any resulting  loss shall be recognized in the Islamic bank’s income statement  loss shall be recognized in the Islamic bank’s income statement  for the financial period in which the loss is realized. (para. 36) for the financial period in which the loss is realized. (para. 36)

%
\subsection{Clause 3/3}%
\label{subsec:Clause3/3}%
Some contemporary Fuqaha are of the view that Istisna’a is valid on  the basis of  the basis of Qiyas Qiyas and the general rules of Shari’a because the fact   and the general rules of Shari’a because the fact  that the subject-matter is non-existent at the time of the constitution  that the subject-matter is non-existent at the time of the constitution  of the contract is compensated for by its preponderant existence  of the contract is compensated for by its preponderant existence  (7) (7) The subject matter may be a commodity, service or both.  The subject matter may be a commodity, service or both. (8) (8) Abdullah, Ahmad Ali, The Juristic Rules of the contract of Istisna’a and Parallel   Abdullah, Ahmad Ali, The Juristic Rules of the contract of Istisna’a and Parallel  Istisna’a, Accounting and Auditing Organization for Islamic Financial Institutions,  Istisna’a, Accounting and Auditing Organization for Islamic Financial Institutions,

%
\subsection{Clause 4/1}%
\label{subsec:Clause4/1}%
Al-Masnoo’

%
\subsection{Clause 4/1/1}%
\label{subsec:Clause4/1/1}%
The revised clause should explicitly include intangible assets as valid subject matter for Istisna'a contracts and provide guidance on how to specify such assets with sufficient clarity to mitigate Gharar, including functional requirements, non-functional requirements, acceptance criteria, and performance metrics.

%
\subsection{Clause 4/1/2}%
\label{subsec:Clause4/1/2}%
The Hanafis stipulate that the commodity contracted for ought  to be of a type of items that people are used to dealing with  to be of a type of items that people are used to dealing with  through Istisna’a. This is important because the legitimacy of  through Istisna’a. This is important because the legitimacy of  Istisna’a is based, according to their viewpoint, on the customary  Istisna’a is based, according to their viewpoint, on the customary  practices of people. practices of people.   However, since the legitimacy of Istisna’a is also based on    However, since the legitimacy of Istisna’a is also based on Qiyas Qiyas, ,  general rules of Shari’a, permissibility of whatever has not been  general rules of Shari’a, permissibility of whatever has not been  considered illegitimate, and  considered illegitimate, and Maslaha Maslaha (consideration of the public   (consideration of the public  good or common need), it is therefore considered a permissible  good or common need), it is therefore considered a permissible  contract to be used whenever the need arises irrespective to  contract to be used whenever the need arises irrespective to  whether or not it has been commonly practised by people. whether or not it has been commonly practised by people.

%
\subsection{Clause 4/1/3}%
\label{subsec:Clause4/1/3}%
Fixing a date for delivering Al-Masnoo’   There are three opinions in the Hanafis School relating to    There are three opinions in the Hanafis School relating to  fixing a date for delivering Al-Masnoo’. fixing a date for delivering Al-Masnoo’. a) Imam Abu Hanifa prevented fixing any future date for the  a) Imam Abu Hanifa prevented fixing any future date for the  delivery of Al-Masnoo’. If a date is fixed, then the contract  delivery of Al-Masnoo’. If a date is fixed, then the contract  turns into Salam because this is a characteristic of a binding  turns into Salam because this is a characteristic of a binding  (10) (10) Al-Darir, M.S.A., op. cit., (P. 466).  Al-Darir, M.S.A., op. cit., (P. 466). (11) (11) Al-Kasani,   Al-Kasani, “Bada`i’ As-Sana`i’ Fi Tartib As-Shara`i’” “Bada`i’ As-Sana`i’ Fi Tartib As-Shara`i’”, Cairo, [5: 2-3]. , Cairo, [5: 2-3]. Financial Accounting Standard No. (10): Istisna’a and Parallel Istisna’a Financial Accounting Standard No. (10): Istisna’a and Parallel Istisna’a

%
\subsection{Clause 4/2}%
\label{subsec:Clause4/2}%
Price   The price should be governed by the following rules:   The price should be governed by the following rules: a) It should be known to the extent of removing ignorance (i.e., lack  a) It should be known to the extent of removing ignorance (i.e., lack  of knowledge). of knowledge). b) It cannot be increased or decreased on account of the normal  b) It cannot be increased or decreased on account of the normal  increase or decrease in commodity prices or cost of labour. increase or decrease in commodity prices or cost of labour. Financial Accounting Standard No. (10): Istisna’a and Parallel Istisna’a Financial Accounting Standard No. (10): Istisna’a and Parallel Istisna’a

%
\subsection{Clause 4/3}%
\label{subsec:Clause4/3}%
The disclosure requirements in Financial Accounting Standard  No. (1): General Presentation and Disclosure in the Financial State- No. (1): General Presentation and Disclosure in the Financial State- ments of Islamic Banks and Financial Institutions should be observed.  ments of Islamic Banks and Financial Institutions should be observed.  (para. 46) (para. 46)

%
\subsection{Clause 5/1}%
\label{subsec:Clause5/1}%
According to the majority of Hanafi jurists, Istisna’a is a valid but not  binding contract. Hence,: binding contract. Hence,:

%
\subsection{Clause 5/1/1}%
\label{subsec:Clause5/1/1}%
Each partner has the option to rescind the contract before  it is implemented. Al-Sani’ has the right not to commence  it is implemented. Al-Sani’ has the right not to commence  manufacturing the goods, while Al-Mustasni’ has the right to  manufacturing the goods, while Al-Mustasni’ has the right to  withdraw from buying Al-Masnoo’. withdraw from buying Al-Masnoo’.

%
\subsection{Clause 5/1/2}%
\label{subsec:Clause5/1/2}%
If Al-Sani’ manufactured Al-Masnoo’, he would not be obliged  to deliver it to Al-Mustasni’. Rather, he has the option to  to deliver it to Al-Mustasni’. Rather, he has the option to  dispense with it in the way he deems fit. This is because the  dispense with it in the way he deems fit. This is because the  contract is not for the manufactured goods themselves, but for  contract is not for the manufactured goods themselves, but for  Al-Masnoo’ of certain specifications. Al-Mustasni’ also has the  Al-Masnoo’ of certain specifications. Al-Mustasni’ also has the  option to accept Al-Masnoo’. option to accept Al-Masnoo’.

%
\subsection{Clause 5/1/3}%
\label{subsec:Clause5/1/3}%
The Hanafis have three different views if Al-Sani’ manufac- tured Al-Masnoo’ according to the specifications and decided  tured Al-Masnoo’ according to the specifications and decided  to deliver it to Al-Mustasni’ in fulfillment of his contractual  to deliver it to Al-Mustasni’ in fulfillment of his contractual  obligations. These are: obligations. These are: a) The preponderant view is that the contract becomes binding  a) The preponderant view is that the contract becomes binding  on Al-Sani’ who has waived his option by delivering Al- on Al-Sani’ who has waived his option by delivering Al- Masnoo’. Yet, the buyer’s option remains to be exercised. This  Masnoo’. Yet, the buyer’s option remains to be exercised. This  view is attributed to the three Imams: Abu Hanifah, Abu  view is attributed to the three Imams: Abu Hanifah, Abu  Yusuf and Muhammad. Yusuf and Muhammad. b) Abu Hanifah is also reported to have said that even at this  b) Abu Hanifah is also reported to have said that even at this  stage Al-Sani’ retains his right on an equal footing with Al- stage Al-Sani’ retains his right on an equal footing with Al- Mustasni’. Mustasni’. c) Abu Yusuf is also reported to have expressed a second opinion  c) Abu Yusuf is also reported to have expressed a second opinion  to the effect that in this situation the contract becomes binding  to the effect that in this situation the contract becomes binding  on the two parties. on the two parties.(12) (12) (12) (12) Al-Sarakhsi, op. cit., [12: 139]; Al-Kasani, op. cit., [5: 3-4]; Al-Babarti,   Al-Sarakhsi, op. cit., [12: 139]; Al-Kasani, op. cit., [5: 3-4]; Al-Babarti, “Al-Inayah  “Al-Inayah ’Ala  Ala  Al-Hidayah” Al-Hidayah”, in  , in “Fath Al-Qadir” “Fath Al-Qadir”, [7: 116]. , [7: 116]. Financial Accounting Standard No. (10): Istisna’a and Parallel Istisna’a Financial Accounting Standard No. (10): Istisna’a and Parallel Istisna’a

%
\subsection{Clause 5/2}%
\label{subsec:Clause5/2}%
The majority in the Hanafi School opined that the Istisna’a contract  is binding once it has been constituted. A number of jurists have  is binding once it has been constituted. A number of jurists have  argued in favour of this view. argued in favour of this view.(13) (13)

%
\subsection{Clause 5/3}%
\label{subsec:Clause5/3}%
Provision (392) of “Majallat Al-Ahkam Al-’Adliyyah” “Majallat Al-Ahkam Al-’Adliyyah” reads as follows:  reads as follows:   Once the contract of Istisna’a is constituted, it becomes binding and    Once the contract of Istisna’a is constituted, it becomes binding and  no party has the right to revoke it. If, however, Al-Masnoo’ does not  no party has the right to revoke it. If, however, Al-Masnoo’ does not  conform to the required specifications, Al-Mustasni’ has the option  conform to the required specifications, Al-Mustasni’ has the option  to revoke the contract. The commentator on the text says: Istisna’a  to revoke the contract. The commentator on the text says: Istisna’a  is a contract of sale and not a mere promise. Once it is constituted,  is a contract of sale and not a mere promise. Once it is constituted,  no party, according to Abu Yousuf’s point of view, has the right to  no party, according to Abu Yousuf’s point of view, has the right to  withdraw unless the consent of the other party is secured (see article  withdraw unless the consent of the other party is secured (see article

%
\subsection{Clause 5/4}%
\label{subsec:Clause5/4}%
In light of the above, all civil legislations based on Shari’a have  treated Istisna’a, in line with the ruling of  treated Istisna’a, in line with the ruling of “Majallat Al-Ahkam Al- “Majallat Al-Ahkam Al- ’Adliyyah” ’Adliyyah”, as a binding contract. These are the Jordanian, Yemeni  , as a binding contract. These are the Jordanian, Yemeni  and Sudanese laws of civil transactions as well as the Unified Arab  and Sudanese laws of civil transactions as well as the Unified Arab  Law proposed by the League of Arab Countries. Law proposed by the League of Arab Countries.

%
\subsection{Clause 5/5}%
\label{subsec:Clause5/5}%
The Islamic Fiqh Academy has also decreed: “The contract of Istisna’a  is binding on its parties provided that certain conditions are fulfilled.” is binding on its parties provided that certain conditions are fulfilled.”   These views strengthen one another and confirm that there is   These views strengthen one another and confirm that there is a substantiated viewpoint in the Hanafi school professing the bind- a substantiated viewpoint in the Hanafi school professing the bind- ing nature of Istisna’a once it is constituted. It is on this reality that  ing nature of Istisna’a once it is constituted. It is on this reality that  the  the “Majallat Al-Ahkam Al-’Adliyyah” “Majallat Al-Ahkam Al-’Adliyyah”, modern civil Islamic legis- , modern civil Islamic legis- lations and the Islamic Fiqh Academy developed their viewpoint,  lations and the Islamic Fiqh Academy developed their viewpoint,  which is consistent with Shari’a rules and principles. which is consistent with Shari’a rules and principles.

%
\subsection{Clause 7/1}%
\label{subsec:Clause7/1}%
Al-Mustasni’ has the right to obtain collateral from Al-Sani’ for: a) The total amount that he has paid. a) The total amount that he has paid. b) The delivery of Al-Masnoo’ in accordance with the specifications  b) The delivery of Al-Masnoo’ in accordance with the specifications  and on due time. and on due time.

%
\subsection{Clause 7/2}%
\label{subsec:Clause7/2}%
Al-Sani’ also has the right to secure collaterals to guarantee that the  price is payable on due time. price is payable on due time.(17) (17)

%
\end{document}