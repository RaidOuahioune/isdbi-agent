\documentclass{article}%
\usepackage[T1]{fontenc}%
\usepackage[utf8]{inputenc}%
\usepackage{lmodern}%
\usepackage{textcomp}%
\usepackage{lastpage}%
%
%
%
\begin{document}%
\normalsize%
\section{AAOIFI Standard (Enhanced)}%
\label{sec:AAOIFIStandard(Enhanced)}%
\subsection{Clause 1/1}%
\label{subsec:Clause1/1}%
Categories of Musharaka   Partnerships are of two types: Holding Partnership and Contract    Partnerships are of two types: Holding Partnership and Contract  Partnership. Partnership.   A holding partnership is created by means of inheritance or wills or    A holding partnership is created by means of inheritance or wills or  other circumstances resulting in the holding by two or more persons  other circumstances resulting in the holding by two or more persons  of an asset in common. In a holding partnership two or more  of an asset in common. In a holding partnership two or more persons  persons  share in a real asset and in the returns arising therefrom. share in a real asset and in the returns arising therefrom.   A contract partnership is created by means of an agreement whereby    A contract partnership is created by means of an agreement whereby  two or more persons agree that each of them contributes to the  two or more persons agree that each of them contributes to the capital  capital  of the partnership and shares in its profit or loss. of the partnership and shares in its profit or loss.(2) (2)   Contract partnerships are divided into: Al-Mufawada, Al-’Inan, Al-   Contract partnerships are divided into: Al-Mufawada, Al-’Inan, Al- A’mal and Al-Wujuh. Fuqaha (jurists) have differed on whether  A’mal and Al-Wujuh. Fuqaha (jurists) have differed on whether  Mudaraba is a partnership in this sense or not. Some Fuqaha consider  Mudaraba is a partnership in this sense or not. Some Fuqaha consider  Mudaraba to be such a partnership because in general it fulfils the  Mudaraba to be such a partnership because in general it fulfils the  elements and terms of a partnership contract. Others, however, do  elements and terms of a partnership contract. Others, however, do  not consider Mudaraba to be one of the types of contract partnership. not consider Mudaraba to be one of the types of contract partnership.   Following is a brief definition of each of the above types in light of    Following is a brief definition of each of the above types in light of  what is reported in Fiqh texts. what is reported in Fiqh texts. (2) (2) Sayyid Sabiq,   Sayyid Sabiq, “Fiqh Al-Sunnah” “Fiqh Al-Sunnah”, [3: 294], (Dar Al-Turath Printing House, Cairo 1977);  , [3: 294], (Dar Al-Turath Printing House, Cairo 1977);  Abdul-Aziz Al-Khayyat,  Abdul-Aziz Al-Khayyat, “Companies in Islamic Shari’a” “Companies in Islamic Shari’a”, 1390 A.H.-1970 A.D., first  , 1390 A.H.-1970 A.D., first  edition, [1: 23, 41 and after].  edition, [1: 23, 41 and after].  Financial Accounting Standard No. (4): Musharaka Financing Financial Accounting Standard No. (4): Musharaka Financing

%
\subsection{Clause 1/1/1}%
\label{subsec:Clause1/1/1}%
Al-’Inan partnership Inan partnership   It is a contract between two or more persons. Each of the parties    It is a contract between two or more persons. Each of the parties  contributes a portion of the overall fund and participates in  contributes a portion of the overall fund and participates in  work. Both parties share in profit or loss as agreed between  work. Both parties share in profit or loss as agreed between  them, but equality is not required either in the contribution to  them, but equality is not required either in the contribution to  the fund or in work or in sharing of profit (these being subject  the fund or in work or in sharing of profit (these being subject  to agreement between the parties). This type of partnership is  to agreement between the parties). This type of partnership is  approved by all Fuqaha. approved by all Fuqaha.   Hanafis and Hanbalis allow any of the followings: Profits of the    Hanafis and Hanbalis allow any of the followings: Profits of the  two parties to be divided in proportion to their contributed  two parties to be divided in proportion to their contributed  funds; profits may be divided equally but contributed funds  funds; profits may be divided equally but contributed funds  may be different; and profits may be unequally divided, but  may be different; and profits may be unequally divided, but  contributed funds are equal. Ibn Qudamah said: “Preference  contributed funds are equal. Ibn Qudamah said: “Preference  in profit is permissible with the existence of work, as one of  in profit is permissible with the existence of work, as one of  the two parties may be more informed as to trade transactions  the two parties may be more informed as to trade transactions  than the other party and/or physically capable of achieving  than the other party and/or physically capable of achieving  greater deal of work, which allows him to make an increase in  greater deal of work, which allows him to make an increase in  his profit  his profit share a condition of his work”. Malikis and Shafis make  share a condition of his work”. Malikis and Shafis make  the  the acceptance of this type of partnership conditional on profits  acceptance of this type of partnership conditional on profits  and losses being proportionate to the size of contributions to  and losses being proportionate to the size of contributions to  the overall fund because (according to them) profit in this type  the overall fund because (according to them) profit in this type  of partnership is considered to be return on capital. of partnership is considered to be return on capital.(3) (3)

%
\subsection{Clause 1/1/2}%
\label{subsec:Clause1/1/2}%
Al-Mufawada partnership   It is a contract between two or more persons. Each of the    It is a contract between two or more persons. Each of the  two parties contributes a portion of the overall fund and  two parties contributes a portion of the overall fund and  participates in work. Both parties equally divide profit or loss.  participates in work. Both parties equally divide profit or loss.  It is a condition of this type of partnership that contributed  It is a condition of this type of partnership that contributed  funds, work, mutual responsibility and liability for debts be  funds, work, mutual responsibility and liability for debts be  equally shared by the parties. Both Hanafis and Malikis have  equally shared by the parties. Both Hanafis and Malikis have  permitted this type of partnership but have stipulated many  permitted this type of partnership but have stipulated many  restrictions for it. restrictions for it.(4) (4) (3) (3) Sayyid Sabiq,   Sayyid Sabiq, “Fiqh Al-Sunnah” “Fiqh Al-Sunnah”, [3: 296]; Abdul-Aziz Al-Khayyat, op. cit., [2: 30-31];  , [3: 296]; Abdul-Aziz Al-Khayyat, op. cit., [2: 30-31];  Al-Kasani,  Al-Kasani, “Bada`i’ Al-Sana`i’ Fi Tartib Al-Shara`ie’” “Bada`i’ Al-Sana`i’ Fi Tartib Al-Shara`ie’”, [6: 57]. , [6: 57]. (4) (4) Al-Kasani, op. cit., (P. 56); Ibn Qudamah,   Al-Kasani, op. cit., (P. 56); Ibn Qudamah, “Al-Mughni” “Al-Mughni”, [6: 30]. , [6: 30]. Financial Accounting Standard No. (4): Musharaka Financing Financial Accounting Standard No. (4): Musharaka Financing

%
\subsection{Clause 1/1/3}%
\label{subsec:Clause1/1/3}%
Al-A’mal partnership   It is a contract between two persons who agree to accept work    It is a contract between two persons who agree to accept work  jointly and to share the profit from such work. For example,  jointly and to share the profit from such work. For example,  two persons of the same profession or craft may agree to work  two persons of the same profession or craft may agree to work  together and to divide the profit arising from such work on  together and to divide the profit arising from such work on  an agreed basis. It is sometimes called Al-Abdan or Al-Sana`i’  an agreed basis. It is sometimes called Al-Abdan or Al-Sana`i’  partnership. partnership.   Al-A’mal partnership is considered permissible by Hanafis, Ma-   Al-A’mal partnership is considered permissible by Hanafis, Ma- likis, Hanbalis. likis, Hanbalis.(5) (5) It is considered valid within the same profes-  It is considered valid within the same profes- sion or otherwise. Its permissibility is based on much evidences,  sion or otherwise. Its permissibility is based on much evidences,  including explicit approval thereon by the Prophet (peace be  including explicit approval thereon by the Prophet (peace be  upon him). In addition, it is based on agency which is permissi- upon him). In addition, it is based on agency which is permissi- ble. This type of partnership has been used throughout without  ble. This type of partnership has been used throughout without  being disapproved of. being disapproved of.(6) (6)

%
\subsection{Clause 1/1/4}%
\label{subsec:Clause1/1/4}%
Al-Wujuh partnership   It is a contract between two or more persons who have good    It is a contract between two or more persons who have good  reputation and prestige and who are expert in trading. Parties  reputation and prestige and who are expert in trading. Parties  to the contract purchase goods on credit from firms, depending  to the contract purchase goods on credit from firms, depending  on their reputation, and sell the goods for cash. They share  on their reputation, and sell the goods for cash. They share  profit or loss according to the guarantee to suppliers provided  profit or loss according to the guarantee to suppliers provided  by each partner. Accordingly, this type of partnership does not  by each partner. Accordingly, this type of partnership does not  require capital since it is based on credit backed by guarantee.  require capital since it is based on credit backed by guarantee.  Hence, it is sometimes called a “Receivables Partnership”. Hence, it is sometimes called a “Receivables Partnership”.   Al-Wujuh partnership is considered permissible by Hanafis    Al-Wujuh partnership is considered permissible by Hanafis  and Hanbalis. Those who support its permissibility argue that  and Hanbalis. Those who support its permissibility argue that  it includes an agency guarantee which is also acceptable. It has  it includes an agency guarantee which is also acceptable. It has  been used throughout without being disapproved of. been used throughout without being disapproved of.(7) (7) (5) (5) Ahmad Ali Abdullah,   Ahmad Ali Abdullah, “Legal Entity in Islamic Fiqh” “Legal Entity in Islamic Fiqh”, (Sudanese Printing Press House,  , (Sudanese Printing Press House,  Khartoum, undated), (pp. 217 and after). Khartoum, undated), (pp. 217 and after). (6) (6) Abdul-Aziz Al-Khayyat, op. cit., [2: 37];. Ibn Qudamah,   Abdul-Aziz Al-Khayyat, op. cit., [2: 37];. Ibn Qudamah, “Al-Mughni” “Al-Mughni”, op. cit. [5: 6];  , op. cit. [5: 6];  Sayyid Sabiq,  Sayyid Sabiq, “Fiqh Al-Sunnah” “Fiqh Al-Sunnah”, op. cit., (P. 297). , op. cit., (P. 297). (7) (7) Abdul-Aziz Al-Khayyat, op. cit., [2: 46-48].  Abdul-Aziz Al-Khayyat, op. cit., [2: 46-48]. Financial Accounting Standard No. (4): Musharaka Financing Financial Accounting Standard No. (4): Musharaka Financing

%
\subsection{Clause 1/2}%
\label{subsec:Clause1/2}%
Musharaka elements and conditions

%
\subsection{Clause 1/2/1}%
\label{subsec:Clause1/2/1}%
Musharaka elements   The elements of Musharaka are: Wording (offer and acceptance),    The elements of Musharaka are: Wording (offer and acceptance),  contract parties (the two contracting parties) and the subject  contract parties (the two contracting parties) and the subject  matter of the agreement (funding and work). matter of the agreement (funding and work).

%
\subsection{Clause 1/2/2}%
\label{subsec:Clause1/2/2}%
Terms of Musharaka

%
\subsection{Clause 1/2/2/1}%
\label{subsec:Clause1/2/2/1}%
Wording   There is no specified form of Musharaka contract. It    There is no specified form of Musharaka contract. It  may be formed by any utterance expressing the purpose.  may be formed by any utterance expressing the purpose.  Contracting is considered to be valid if made verbally  Contracting is considered to be valid if made verbally  or in writing. The Musharaka contract is notarised in  or in writing. The Musharaka contract is notarised in  writing and witnessed. writing and witnessed.

%
\subsection{Clause 1/2/2/2}%
\label{subsec:Clause1/2/2/2}%
Contracting parties   It is a requirement that the partner should be competent    It is a requirement that the partner should be competent  to give or be given power of attorney. to give or be given power of attorney.

%
\subsection{Clause 1/2/2/3}%
\label{subsec:Clause1/2/2/3}%
Subject matter of the contract (funding and work)   There are the following requirements:   There are the following requirements: a) Funding a) Funding   Capital contributed shall be in cash, gold, silver or    Capital contributed shall be in cash, gold, silver or  their equivalent in value. There is no difference   their equivalent in value. There is no difference   among Fuqaha in this respect. among Fuqaha in this respect.   Capital may consist of trading assets such as goods,    Capital may consist of trading assets such as goods,  property and equipment, etc. It may also be in the  property and equipment, etc. It may also be in the  form of intangible rights, such as liens, patents and  form of intangible rights, such as liens, patents and  suchlike. It is considered permissible by some Fuqaha  suchlike. It is considered permissible by some Fuqaha  that the capital of a company can be contributed in  that the capital of a company can be contributed in  the form of these types of assets provided they are  the form of these types of assets provided they are  valued at their cash equivalent according to what the  valued at their cash equivalent according to what the  partners agree upon. partners agree upon.   Shafis and Malikis argue that the funds provided    Shafis and Malikis argue that the funds provided  by partners should be commingled in order that  by partners should be commingled in order that  Financial Accounting Standard No. (4): Musharaka Financing Financial Accounting Standard No. (4): Musharaka Financing

%
\subsection{Clause 1/3}%
\label{subsec:Clause1/3}%
Musharaka rules

%
\subsection{Clause 1/3/1}%
\label{subsec:Clause1/3/1}%
Rules of capital   Following are the most significant rules which control the    Following are the most significant rules which control the  operation of capital and its maintenance: operation of capital and its maintenance: a) Power of attorney and disposition of funds a) Power of attorney and disposition of funds  Any partner has the right to dispose of the partnership’s   Any partner has the right to dispose of the partnership’s  assets in the normal course of business. A partnership  assets in the normal course of business. A partnership with  with  a contributed capital (e.g., Al-’Inan) constitutes an entity  a contributed capital (e.g., Al-’Inan) constitutes an entity  and once the capital has been contributed, it comprises a  and once the capital has been contributed, it comprises a  single fund. Each partner empowers his other partner(s)  single fund. Each partner empowers his other partner(s)  to dispose of the assets and he is thus considered to be  to dispose of the assets and he is thus considered to be  authorised to employ them in the activity of the Musharaka  authorised to employ them in the activity of the Musharaka  provided he does so with due care to the interests of his  provided he does so with due care to the interests of his  partner(s) and without misconduct or negligence. A  partner(s) and without misconduct or negligence. A partner  partner  is not allowed to disburse or invest the funds for his  is not allowed to disburse or invest the funds for his personal  personal  purposes. purposes. b) Non-guarantee of capital b) Non-guarantee of capital  Neither partner can guarantee the other partner’s capital,   Neither partner can guarantee the other partner’s capital,  because Musharaka is based on the principle of Al-Ghurm  because Musharaka is based on the principle of Al-Ghurm  (8) (8) Ibn Qudamah,   Ibn Qudamah, “Al-Mughni” “Al-Mughni”, op. cit., [5: 17]. , op. cit., [5: 17]. Financial Accounting Standard No. (4): Musharaka Financing Financial Accounting Standard No. (4): Musharaka Financing

%
\subsection{Clause 1/3/2}%
\label{subsec:Clause1/3/2}%
Work rules   In a partnership with a contributed capital, the partners    In a partnership with a contributed capital, the partners shall  shall  provide both funds and work, and each partner shall  provide both funds and work, and each partner shall undertake  undertake  work as an agent of the partnership subject to the partnership  work as an agent of the partnership subject to the partnership  contract. This is regulated by a number of juristic rules, the  contract. This is regulated by a number of juristic rules, the  most significant of which are: most significant of which are: a) Agency as to the work  a) Agency as to the work   Each partner carries out work in the partnership on behalf of   Each partner carries out work in the partnership on behalf of  himself and as an agent for his partner. This is governed by  himself and as an agent for his partner. This is governed by  the general rules of agency contract in Islamic jurisprudence.  the general rules of agency contract in Islamic jurisprudence.  Some of these rules are related to the principal and others are  Some of these rules are related to the principal and others are  related to the agent and some are related to the things which  related to the agent and some are related to the things which  are the subject of agency. All these matters should be made  are the subject of agency. All these matters should be made  clear in the Musharaka contract. clear in the Musharaka contract.(9) (9) b) Scope of the work b) Scope of the work  This relates to the specification of the scope of each partner’s   This relates to the specification of the scope of each partner’s  work in the partnership in relation to the latter’s objectives  work in the partnership in relation to the latter’s objectives  and activities. The partner should perform the agreed  and activities. The partner should perform the agreed work  work  without negligence or misconduct. Partnership work  without negligence or misconduct. Partnership work in- in- cludes management of the business (e.g., planning, policy  cludes management of the business (e.g., planning, policy  making, development of executive programs, following-up,  making, development of executive programs, following-up,  supervision, performance appraisal and decision-making).  supervision, performance appraisal and decision-making).  (9) (9) Ibn Qudamah,   Ibn Qudamah, “Al-Mughni” “Al-Mughni”, [5: 88 and after], Kitab Al-Wakalah; Sayyid Sabiq,  , [5: 88 and after], Kitab Al-Wakalah; Sayyid Sabiq, “Fiqh  “Fiqh  Al-Sunnah” Al-Sunnah”, [3: 226], (Agency Chapter).  , [3: 226], (Agency Chapter).  Financial Accounting Standard No. (4): Musharaka Financing Financial Accounting Standard No. (4): Musharaka Financing

%
\subsection{Clause 1/3/3}%
\label{subsec:Clause1/3/3}%
Rules of profit a) General rules of profit a) General rules of profit(12) (12)

%
\subsection{Clause 1/3/4}%
\label{subsec:Clause1/3/4}%
Rules applicable in case of loss   Fuqaha agree that loss should be divided between the partners    Fuqaha agree that loss should be divided between the partners  in proportion to their respective shares in the capital. Fuqaha  in proportion to their respective shares in the capital. Fuqaha  call this “Wadhi’ah” (loss). They support this opinion by the  call this “Wadhi’ah” (loss). They support this opinion by the  following saying of Ali Ibn Abu Talib (may Allah be pleased  following saying of Ali Ibn Abu Talib (may Allah be pleased  with him): “Profit should be according to what they (partners)  with him): “Profit should be according to what they (partners)  (14) (14) Refer to: Ibn Rushd,   Refer to: Ibn Rushd, “Bidayat Al-Mujtahid Wa Nihayat Al-Muqtasid” “Bidayat Al-Mujtahid Wa Nihayat Al-Muqtasid”, [2: 253];  , [2: 253];  Al-Khatib,  Al-Khatib, “Mughni Al-Muhtaj Sharh Al-Minhaj” “Mughni Al-Muhtaj Sharh Al-Minhaj”, [2: 215], (Dar Ihya` Al-Turath  , [2: 215], (Dar Ihya` Al-Turath  Al-Arabi, Beirut); Ibn Qudamah,  Al-Arabi, Beirut); Ibn Qudamah, “Al-Mughni” “Al-Mughni”, [5: 30 and 31]; Mahmud Ibn Ahmad  , [5: 30 and 31]; Mahmud Ibn Ahmad  Al-’Ayni  Al-’Ayni “Al-Binayah Fi Sharh Al-Hidayah” “Al-Binayah Fi Sharh Al-Hidayah”, [6: 108].  , [6: 108].  Financial Accounting Standard No. (4): Musharaka Financing Financial Accounting Standard No. (4): Musharaka Financing

%
\subsection{Clause 1/3/5}%
\label{subsec:Clause1/3/5}%
Rules of Musharaka termination   In general, the partnership shall be terminated if one of the    In general, the partnership shall be terminated if one of the  partners terminates the contract, or dies, if his legal competency  partners terminates the contract, or dies, if his legal competency  ceases or if the partnership capital is lost. ceases or if the partnership capital is lost.   The majority of Fuqaha, except for Malikis, are of the opinion    The majority of Fuqaha, except for Malikis, are of the opinion  that as partnership is one of the permissible forms of contract,  that as partnership is one of the permissible forms of contract,  each of the partners is entitled to terminate it whenever he  each of the partners is entitled to terminate it whenever he  wishes, as is the case with agency contracts. wishes, as is the case with agency contracts.   The partnership is based on agency and probity. Each of the    The partnership is based on agency and probity. Each of the  partners is a proxy for the others and a principal at the same  partners is a proxy for the others and a principal at the same  time. He acts in respect of his share as a principal and in respect  time. He acts in respect of his share as a principal and in respect  of his partners’ shares as a proxy; i.e., as an agent. In principle,  of his partners’ shares as a proxy; i.e., as an agent. In principle,  agency is one of the unanimously permissible contracts and  agency is one of the unanimously permissible contracts and  no party is forced to proceed with it against his will. The  no party is forced to proceed with it against his will. The  partnership, as well, should start with an agency relationship  partnership, as well, should start with an agency relationship  between the partners, and this relationship provides the basis  between the partners, and this relationship provides the basis  for its continuity. If the agency relationship is severed by  for its continuity. If the agency relationship is severed by  termination on the part of one of the partners, the legal basis  termination on the part of one of the partners, the legal basis  upon which they acted in respect of each other’s funds will be  upon which they acted in respect of each other’s funds will be  eliminated. eliminated.(16) (16)   In the case of death, one of the heirs, if he is of sound mind,    In the case of death, one of the heirs, if he is of sound mind,  may replace the deceased provided that the other heirs and  may replace the deceased provided that the other heirs and  the other partners agree to that. This shall also be applicable  the other partners agree to that. This shall also be applicable in  in  case one of the partners loses competency. case one of the partners loses competency. (15) (15) Al-’Ayni,   Al-’Ayni, “Al-Binayah Fi Sharh Al-Hidayah” “Al-Binayah Fi Sharh Al-Hidayah”, op. cit., [6: 108]; Ibn Qudamah,  , op. cit., [6: 108]; Ibn Qudamah, “Al-Mughni” “Al-Mughni”, ,  [5: 37], The Case of : Wadi’ah Should Be Proportionate to the Amount of Fund.  [5: 37], The Case of : Wadi’ah Should Be Proportionate to the Amount of Fund.  (16) (16) Ali Al-Khafif,   Ali Al-Khafif, “Companies in Islamic Jurisprudence” “Companies in Islamic Jurisprudence”, op. cit., (P. 548).  , op. cit., (P. 548).  Financial Accounting Standard No. (4): Musharaka Financing Financial Accounting Standard No. (4): Musharaka Financing

%
\subsection{Clause 2/1}%
\label{subsec:Clause2/1}%
Recognition of the Islamic bank’s share in Musharaka capital at the  time of contracting time of contracting   The Islamic bank’s share in Musharaka capital (cash or kind) shall    The Islamic bank’s share in Musharaka capital (cash or kind) shall  be recognized when it is paid to the partner or made available to him  be recognized when it is paid to the partner or made available to him  on the account of the Musharaka. This share shall be presented in  on the account of the Musharaka. This share shall be presented in  the Islamic bank’s books under a Musharaka financing account with  the Islamic bank’s books under a Musharaka financing account with  (name of client) and it shall be included in the financial statements  (name of client) and it shall be included in the financial statements  under the heading “Musharaka Financing”. (para. 3) under the heading “Musharaka Financing”. (para. 3) Financial Accounting Standard No. (4): Musharaka Financing Financial Accounting Standard No. (4): Musharaka Financing

%
\subsection{Clause 2/2}%
\label{subsec:Clause2/2}%
Measurement of the Islamic bank’s share in Musharaka capital at the  time of contracting time of contracting

%
\subsection{Clause 2/2/1}%
\label{subsec:Clause2/2/1}%
The Islamic bank’s share in the Musharaka capital provided in  cash shall be measured by the amount paid or made available  cash shall be measured by the amount paid or made available  to the partner on the account of the Musharaka. (para. 4) to the partner on the account of the Musharaka. (para. 4)

%
\subsection{Clause 2/2/2}%
\label{subsec:Clause2/2/2}%
The Islamic bank’s share in Musharaka capital provided in kind  kind  (trading assets or non-monetary assets for use in the venture)  (trading assets or non-monetary assets for use in the venture)  shall be measured at the fair value of the assets (the value  shall be measured at the fair value of the assets (the value  agreed between the partners), and if the valuation of the assets  agreed between the partners), and if the valuation of the assets  results in a difference between fair value and book value, such  results in a difference between fair value and book value, such  difference shall be recognized as profit or loss to the Islamic  difference shall be recognized as profit or loss to the Islamic  bank itself. (para. 5) bank itself. (para. 5)

%
\subsection{Clause 2/2/3}%
\label{subsec:Clause2/2/3}%
Expenses of the contracting procedures incurred by one or both  parties (e.g., expenses of feasibility studies and other  parties (e.g., expenses of feasibility studies and other similar  similar  expenses) shall not be considered as part of the Musharaka  expenses) shall not be considered as part of the Musharaka  capital unless otherwise agreed by both parties. (para. 6) capital unless otherwise agreed by both parties. (para. 6)

%
\subsection{Clause 2/3}%
\label{subsec:Clause2/3}%
Measurement of the Islamic bank’s share in Musharaka capital after  contracting at the end of a financial period contracting at the end of a financial period

%
\subsection{Clause 2/3/1}%
\label{subsec:Clause2/3/1}%
The Islamic bank’s share in the constant Musharaka capital  shall be measured at the end of the financial period at  shall be measured at the end of the financial period at historical  historical  cost (the amount which was paid or at which the asset was  cost (the amount which was paid or at which the asset was  valued at the time of contracting). (para. 7) valued at the time of contracting). (para. 7)

%
\subsection{Clause 2/3/2}%
\label{subsec:Clause2/3/2}%
The Islamic bank’s share in the diminishing Musharaka shall  be measured at the end of a financial period at historical cost  be measured at the end of a financial period at historical cost  after deducting the historical cost of any share transferred  after deducting the historical cost of any share transferred to the  to the  partner (such transfer being by means of a sale at fair value).  partner (such transfer being by means of a sale at fair value).  The difference between historical cost and fair value shall  The difference between historical cost and fair value shall  be recognized as profit or loss in the Islamic bank’s income  be recognized as profit or loss in the Islamic bank’s income  statement. (para. 8) statement. (para. 8)

%
\subsection{Clause 2/3/3}%
\label{subsec:Clause2/3/3}%
If the diminishing Musharaka is liquidated before complete  transfer is made to the partner, the amount recovered in respect transfer is made to the partner, the amount recovered in respect  Financial Accounting Standard No. (4): Musharaka Financing Financial Accounting Standard No. (4): Musharaka Financing

%
\subsection{Clause 2/3/4}%
\label{subsec:Clause2/3/4}%
If the Musharaka is terminated or liquidated and the Islamic  bank’s due share of the Musharaka capital (taking account of  bank’s due share of the Musharaka capital (taking account of  any profits or losses) remains unpaid when a settlement of  any profits or losses) remains unpaid when a settlement of  account is made, the Islamic bank’s share shall be recognized  account is made, the Islamic bank’s share shall be recognized  as a receivable due from the partner. (para. 10) as a receivable due from the partner. (para. 10)

%
\subsection{Clause 2/4}%
\label{subsec:Clause2/4}%
Recognition of the Islamic bank’s share in Musharaka profits or  losses losses

%
\subsection{Clause 2/4/1}%
\label{subsec:Clause2/4/1}%
Profits or losses in respect of Musharaka transactions should be recognized immediately upon agreement, not just at liquidation. This enhances transparency and compliance with Shariah accountability.

%
\subsection{Clause 2/4/2}%
\label{subsec:Clause2/4/2}%
In the case of a constant Musharaka that continues for more  than one financial period, the Islamic bank’s share of profits  than one financial period, the Islamic bank’s share of profits  for any period, resulting from partial or final settlement be- for any period, resulting from partial or final settlement be- tween the Islamic bank and the partner, shall be recognized  tween the Islamic bank and the partner, shall be recognized  in its accounts for that period to the extent that the profits  in its accounts for that period to the extent that the profits  are being distributed; the Islamic bank’s share of losses for  are being distributed; the Islamic bank’s share of losses for  any period shall be recognized in its accounts for that period  any period shall be recognized in its accounts for that period  to the extent that such losses are being deducted from its  to the extent that such losses are being deducted from its share  share  of the Musharaka capital. (para. 12) of the Musharaka capital. (para. 12)

%
\subsection{Clause 2/4/3}%
\label{subsec:Clause2/4/3}%
Item 2/4/2 shall apply to a diminishing Musharaka which  continues for more than one financial period, after taking  continues for more than one financial period, after taking  into consideration the decline in the Islamic bank’s share in  into consideration the decline in the Islamic bank’s share in  Musharaka capital and its profits or losses. (para. 13) Musharaka capital and its profits or losses. (para. 13)

%
\subsection{Clause 2/4/4}%
\label{subsec:Clause2/4/4}%
As implied by item 2/3/4 above, if the partner does not pay the  Islamic bank its due share of profits after liquidation or after  Islamic bank its due share of profits after liquidation or after  settlement of account is made, the due share of profits shall be  settlement of account is made, the due share of profits shall be  recognized as a receivable due from the partner. (para. 14) recognized as a receivable due from the partner. (para. 14) Financial Accounting Standard No. (4): Musharaka Financing Financial Accounting Standard No. (4): Musharaka Financing

%
\subsection{Clause 2/4/5}%
\label{subsec:Clause2/4/5}%
If losses are incurred in a Musharaka due to the partner’s  misconduct or negligence, the partner shall bear the Islamic  misconduct or negligence, the partner shall bear the Islamic  bank’s share of such losses. Such losses shall be recognized as  bank’s share of such losses. Such losses shall be recognized as  a receivable due from the partner. (para. 15) a receivable due from the partner. (para. 15)

%
\subsection{Clause 2/4/6}%
\label{subsec:Clause2/4/6}%
The Islamic bank’s unpaid share of the proceeds as men- tioned  tioned above in items 2/3/4 and 2/4/4 shall be recorded in above in items 2/3/4 and 2/4/4 shall be recorded in a Musharaka receivables account. A provision shall be made  a Musharaka receivables account. A provision shall be made  for these receivables if they are doubtful. (para. 16) for these receivables if they are doubtful. (para. 16)

%
\subsection{Clause 2/5}%
\label{subsec:Clause2/5}%
Disclosure requirements

%
\subsection{Clause 2/5/1}%
\label{subsec:Clause2/5/1}%
Disclosure should be made in the notes to the financial state- ments for a financial reporting period if the Islamic bank has  ments for a financial reporting period if the Islamic bank has  made during that period a provision for a loss of its capital in  made during that period a provision for a loss of its capital in  Musharaka financing transactions. (para. 17) Musharaka financing transactions. (para. 17)

%
\subsection{Clause 2/5/2}%
\label{subsec:Clause2/5/2}%
The disclosure requirements should explicitly include Shariah review board comments and compliance notes in the financial statements.

%
\end{document}